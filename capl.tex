% Created 2018-09-14 Fri 11:30
% Intended LaTeX compiler: pdflatex
\documentclass[11pt]{article}
\usepackage[utf8]{inputenc}
\usepackage[T1]{fontenc}
\usepackage{graphicx}
\usepackage{grffile}
\usepackage{longtable}
\usepackage{wrapfig}
\usepackage{rotating}
\usepackage[normalem]{ulem}
\usepackage{amsmath}
\usepackage{textcomp}
\usepackage{amssymb}
\usepackage{capt-of}
\usepackage{hyperref}
\title{Crazy Document}
\author{Yiping Deng}
\date{\today}
\title{}
\hypersetup{
 pdfauthor={Yiping Deng},
 pdftitle={},
 pdfkeywords={},
 pdfsubject={},
 pdfcreator={Emacs 26.1 (Org mode 9.1.14)}, 
 pdflang={'en'}}
\begin{document}

\tableofcontents
\maketitle
\section{Info}
\label{sec:org800c109}
course website: \url{https://grader.eecs.jacobs-university.de/courses/320241/2018\_2/} \\
office hours: Mondays 10:00 - 12:00 \\
there are two TAs, they will hold tutorials. \\
\subsection{books:}
\label{sec:org460b20a}
\begin{itemize}
\item digital systems
\item computer organization and design
\item Compilers: Principles, Techniques, and Tools(Dragon book)
\item Compiler Design in C
\end{itemize}
\subsection{goals:}
\label{sec:org09cee48}
understand the basic knowledge of:
\begin{itemize}
\item computer architecture
\item data representation
\item instruction set architectures
\item datapath and control
\begin{itemize}
\item why are logic gate important, how to optimize
\end{itemize}
\item programming languages characteristics
\item phases of compilation
\end{itemize}
\subsection{grading scheme:}
\label{sec:orgbd94059}
\begin{itemize}
\item 30\% homework(in total 12)
\item 30\% midterm
\item 40\% final
\end{itemize}
\subsection{homework:}
\label{sec:orgc72b31f}
\begin{itemize}
\item individual submission
\item homework should be submitted in a single pdf, need a \href{https://grader.eecs.jacobs-university.de/courses/320241/2018\_2/lectures/template\_hw.tex}{homework template}
\item homework is due on Tuesday 14:00 sharp
\end{itemize}
\subsubsection{grading criteria}
\label{sec:org964fc9d}
\begin{itemize}
\item 10\% homeowork formatting
\item 50- 70\% intermediate steps
\end{itemize}
\subsection{tutorials:}
\label{sec:orga23d9b4}
\begin{itemize}
\item Sundays in West Hall 4 at 17:00
\end{itemize}
\section{Introduction}
\label{sec:org433aa70}
\subsection{history}
\label{sec:org092c7a2}
\begin{enumerate}
\item mechanical machines
\item electronic mchines
\item digital computers
\item networking
\end{enumerate}
\subsection{processor}
\label{sec:orgc0fcbcc}
\begin{itemize}
\item Pentium I: 60MHz, 800 nm
\item Nehalem: 45 nm
\item Gulfdown: 32 nm
\end{itemize}
\subsection{numerical representation and numerical system}
\label{sec:org7ea9f90}
\subsubsection{analog v. digital}
\label{sec:orga9c06c3}
physical systems use quantities which must be manipulated
arithmetically. Quantities may be represented numerically in either analog or
digital form.
\begin{itemize}
\item analog: a continuous variable, proportional indicator
\item digital: varies in discrete step
\end{itemize}
\begin{table}[htbp]
\caption{Digital v. Analog}
\centering
\begin{tabular}{ll}
Digital & Analog\\
\hline
discrete steps & a continuous variable\\
ease of design & \\
well suited for storing information & \\
accurate and easy to maintain & \\
programmable operation & \\
less affected by noise & \\
ease of fabrication & \\
\end{tabular}
\end{table}
\subsubsection{digital number systems}
\label{sec:org9d99af2}
\begin{itemize}
\item decimal systems
\begin{itemize}
\item 10 symbols: 0, 1, 2, 3, 4, 5, 6, 7, 8, 9
\item each number is a digit
\item most significant digit (MSD) and least significant digit (LSD)
\item Positional value may be stated as a digit multiplied by a power of 10
\end{itemize}
\item binary systems
\begin{itemize}
\item lends itselfs to electronic circuit design
\item can convert to other representation
\item Positional value may be stated as a digit multiplied by a power of 2
\end{itemize}
\end{itemize}
\subsubsection{digital circuit v. logic circuit}
\label{sec:org2fca257}
\begin{itemize}
\item Digital circuits – produce and respond to predefined voltage ranges
\item Logic circuits – used interchangeably with the term, digital circuits
\item Digital integrated circuits (ICs) – provide logic operations in a small reliable package
\end{itemize}
\subsubsection{parallel v. serial}
\label{sec:org8469d59}
\begin{itemize}
\item Parallel transmission – all bits in a binary number are transmitted simultaneously. A separate line is required for each bit
\item Serial transmission – each bit in a binary number is transmitted per some time interval
\item Both methods have useful applications which will be seen in later chapters
\end{itemize}
\subsubsection{memory}
\label{sec:org1d5ecf5}
\begin{itemize}
\item A circuit which retains a response to a momentary input is displaying memory
\item Memory is important because it provides a way to store binary numbers temporarily or permanently
\item Memory elements include:
\begin{itemize}
\item Magnetic
\item Optical
\item Electronic latching circuits
\end{itemize}
\end{itemize}
\section{Digital system}
\label{sec:org3a04e90}
\subsection{Conversion}
\label{sec:orga0ea7f1}
\subsubsection{Binary system}
\label{sec:orga747be8}
\begin{enumerate}
\item Binary to decimal conversion
\label{sec:org0814c6c}
Just simply raise 2 to the corresponding power and time the digit from right to the left
\item Decimal to binary conversion
\label{sec:org45df9e2}
Repeated integer division. We continuously divide the number by 2, note down all the reminder, and then reverse the order
\end{enumerate}
\subsubsection{Hexadecimal Number system}
\label{sec:org7edd546}
\begin{itemize}
\item base 16, 16 possible symbols, 0 - 9 and A - F
\item uses groups of 4 bits
\item allows for convenient handling of long binary strings
\end{itemize}
\begin{enumerate}
\item hexadecimal to decimal
\label{sec:org33e2bb4}
163\(_{\text{16}}\) = 1 * (16\(^{\text{2}}\)) + 6 * (16\(^{\text{1}}\)) + 3 * (16\(^{\text{0}}\))
\item decimal to hexadecimal
\label{sec:org90bf14c}
still using repeated integer division
\item hexadecimal to binary
\label{sec:orge1bc293}
use the conversion table
\item binary to hexadecimal
\label{sec:org18753b8}
note that we group 4 bits from the right, fill in 0 on the left side if the number of digits are not divisiable by 4
\end{enumerate}
\subsubsection{Binary-coded decimal(BCD)}
\label{sec:orgf1de88c}
\begin{itemize}
\item each decimal digits are converted to binary accordingly.
\item Programmable calculators manufactured by Texas Instruments.
\item it is not a number system
\end{itemize}
\subsubsection{Bytes, Nibbles and Words}
\label{sec:orgf69edf4}
\begin{itemize}
\item 1 byte = 8 bits
\item 1 nibble = 4 bits
\item 1 word = depending on a data bus width
\begin{itemize}
\item x86: 32 bits
\item arm64: 64bits
\end{itemize}
\end{itemize}
\subsubsection{ASCII code}
\label{sec:org9e5d467}
\begin{itemize}
\item 0 - 127
\end{itemize}
\subsection{Logic Functions}
\label{sec:org9d88606}
\subsubsection{boolean constants and variables}
\label{sec:org316d2f1}
\begin{itemize}
\item Boolean algebra allows only value 0 and 1
\item Logic 0 can be: false / off / low / open switch
\item Logic 1 can be: true / on / high / close switch
\item three basic logic operators:
\begin{itemize}
\item OR
\item AND
\item NOT
\end{itemize}
\end{itemize}
\subsubsection{to generate truth table}
\label{sec:org44654ed}
\begin{verbatim}
from itertools import zip_longest as izip, product, tee

# Logic functions: take and return iterators of truth values

def AND(a, b):
    for p, q in izip(a, b):
        yield p and q

def OR(a, b):
    for p, q in izip(a, b):
        yield p or q

def NOT(a):
    for p in a:
        yield not p

def EQUIV(a, b):
    for p, q in izip(a, b):
        yield p is q

def IMPLIES(a, b):
    for p, q in izip(a, b):
        yield (not p) or q

def create(num=2):
    ''' Returns a list of all of the possible combinations of truth for the given number of variables.
    ex. [(T, T), (T, F), (F, T), (F, F)] for two variables '''
    return list(product([True, False], repeat=num))

def unpack(data):
    ''' Regroups the list returned by create() by variable, making it suitable for use in the logic functions.
    ex. [(T, T, F, F), (T, F, T, F)] for two variables '''
    return [[elem[i] for elem in lst] for i, lst in enumerate(tee(data, len(data[0])))]

def print_result(data, result):
    ''' Prints the combinations returned by create() and the results returned by the logic functions in a nice format. '''
    n = len(data[0])
    headers = 'ABCDEFGHIJKLMNOPQRSTUVWXYZ'[:n]
    begin_format = '|'.join(n * 'c')
    begin_str = '\\begin{tabular}{|' + begin_format + '|c|}'
    print(begin_str)
    print('\\hline')
    print(' & '.join(headers) + ' & Result \\\\')
    print('\\hline')
    for row, result_cell in izip(data, result):
        print(' & '.join({True: 'T', False:'F'}[cell] for cell in row) + ' &' + ' ' + {True: 'T', False:'F'}[result_cell] + ' \\\\')
    print('\\hline')
    print('\\end{tabular}')
data = create(num=4)
A, B, C, D= unpack(data)
result = OR(OR(OR(A, B), C), D)
print_result(data, result)
\end{verbatim}
\subsection{Logic Circuits}
\label{sec:orgd226282}
\subsubsection{Minimization of Logic Expression: rules}
\label{sec:org2e5d079}
\begin{itemize}
\item \(w(y + z) = wy + wz\)   (distributive rule)
\item \(w + \overline{w} = 1\)
\end{itemize}
\end{document}