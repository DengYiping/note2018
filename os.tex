% Created 2018-09-13 Thu 17:29
% Intended LaTeX compiler: pdflatex
\documentclass[11pt]{article}
\usepackage[utf8]{inputenc}
\usepackage[T1]{fontenc}
\usepackage{graphicx}
\usepackage{grffile}
\usepackage{longtable}
\usepackage{wrapfig}
\usepackage{rotating}
\usepackage[normalem]{ulem}
\usepackage{amsmath}
\usepackage{textcomp}
\usepackage{amssymb}
\usepackage{capt-of}
\usepackage{hyperref}
\author{Yiping Deng}
\date{\today}
\title{}
\hypersetup{
 pdfauthor={Yiping Deng},
 pdftitle={},
 pdfkeywords={},
 pdfsubject={},
 pdfcreator={Emacs 26.1 (Org mode 9.1.14)}, 
 pdflang={'en'}}
\begin{document}

\tableofcontents

\section{Info}
\label{sec:org90fee3a}
\begin{itemize}
\item course website: \href{https://cnds.jacobs-university.de/courses/os-2018/}{website}
\item slides: \href{./os\_slides.pdf\#20}{slides}
\item notes: \href{./os\_notes.pdf}{notes}
\end{itemize}
\subsection{grading scheme}
\label{sec:orgea74e9c}
\begin{itemize}
\item final exam 40\%
\item quizzes 30\%
\begin{itemize}
\item 6 quizzes, with 10 pts each
\end{itemize}
\item assignment 30\%
\end{itemize}
\section{Tools}
\label{sec:org6bfe085}
\begin{itemize}
\item ltrace: to trace the library call
\item strace: to trace the system call
\item ldd: to show the library used in the programs
\item insmod: to insert program into the kernel
\item /dev/null: abandon output
\item /dev/full: cannot write
\item /dev/random: random file
\item pmap: to show the memory information for a specific program
\end{itemize}
\section{Intro}
\label{sec:org0fb52e5}
\subsection{system call v. library call}
\label{sec:org5270b76}
\begin{itemize}
\item system call is expensive
\end{itemize}
\subsection{requirement of OS}
\label{sec:org07bc1a0}
\begin{itemize}
\item An operating system should manage resources in a way that avoids shortages or overload conditions
\item An operating system should be efficient and introduce little overhead
\item An operating system should be robust against malfunctioning application
\end{itemize}
programs
\begin{itemize}
\item Data and programs should be protected against unauthorized access and hardware failures
\end{itemize}
\subsection{services for application}
\label{sec:org9a255c1}
\begin{itemize}
\item Loading of programs
\item Execution of programs (management of processes)
\item High-level input/output operations
\item Logical file systems (open(), write(), \ldots{})
\item Control of peripheral devices
\item Interprocess communication primitives
\item Support of basic communication protocols
\item Checkpoint and restart primitives
\item \ldots{}
\end{itemize}
\subsection{services for system operations}
\label{sec:orgb5bea22}
\begin{itemize}
\item User identification and authentication
\item Access control mechanisms
\item Support for cryptographic operations and the management of keys - Control functions (e.g., forced abort of processes)
\item Testing and repair functions (e.g., file systems checks)
\item Monitoring functions (observation of system behavior)
\item Logging functions (collection of event logs)
\item Accounting functions (collection of usage statistics)
\begin{itemize}
\item useful for cloud application
\end{itemize}
\item System generation and system backup functions
\item Software management functions
\item \ldots{}
\end{itemize}
\subsection{to trace the library call}
\label{sec:orgf8dfd69}
`ltrace`
\begin{itemize}
\item our hello worlds appers to use `putc`, it is done in the compile time
\end{itemize}
\subsection{to trace the system call}
\label{sec:org63f0631}
`strace`
\begin{itemize}
\item it produces write(1, "hello, world")
\end{itemize}
\subsection{to show library used in the problem}
\label{sec:org506a7ff}
`ldd`
\begin{itemize}
\item if you use compiler flag -static, it will produces a bigger executable.
\end{itemize}
\subsection{printf v. write}
\label{sec:orgc7d84be}
\begin{itemize}
\item use printf, it is buffered. Using the buffer we can reduce the \# of system call.
\end{itemize}
\subsection{return value of the main function}
\label{sec:org4a3ace2}
Should be 0 if success, other for failure
\subsection{what happened in the hello world program}
\label{sec:orgf1538b5}
\begin{itemize}
\item c library: `printf()`, `puts()`, `fflush(), all this function will make a system call `write()`
\begin{itemize}
\item \textbf{user mode}
\end{itemize}
\item OS kernel: sys\(_{\text{write}}\)()
\begin{itemize}
\item \textbf{kernel mode} \footnote{crossing the boundry between user mode and kernel mode is expensive}
\item in the kernel, there is a table indexed by the system call number
\end{itemize}
\end{itemize}
\subsection{kernel space}
\label{sec:orgb3cba28}
\begin{itemize}
\item the equivalent of `printf` is `printk`
\item running program in the keneral is different from running program in the user space
\item to push a module into the kernel, use `sudo insmod hello/hello.ko`
\item kernel have the highest priviledge, running on Ring 0
\end{itemize}
\subsection{types of operating system}
\label{sec:orgcd7345a}
\subsubsection{Batch Processing Operating Systems}
\label{sec:org5b391ae}
\begin{itemize}
\item Characteristics:
\begin{itemize}
\item Batch jobs are processed sequentially from a job queue
\item Job inputs and outputs are saved in files or printed
\item No interaction with the user during the execution of a batch program
\end{itemize}
\item Batch processing operating systems were the early form of operating systems.
\item Batch processing functions still exist today, for example to execute jobs on super computers.
\end{itemize}
\subsubsection{General Purpose Operating Systems}
\label{sec:org32b8cd9}
\begin{itemize}
\item Characteristics:
\begin{itemize}
\item Multiple programs execute simultaneously (multi-programming, multi-tasking)
\item Multiple users can use the system simultaneously (multi-user)
\item Processor time is shared between the running processes (time-sharing)
\item Input/output devices operate concurrently with the processors
\item Network support but no or very limited transparency
\end{itemize}
\item Examples:
\begin{itemize}
\item Linux, BSD, Solaris
\item Windows, MacOS
\end{itemize}
\end{itemize}
\subsubsection{Parallel Operating Systems}
\label{sec:orgc003bfb}
\begin{itemize}
\item Characteristics:
\begin{itemize}
\item Support for a very large number of tightly integrated processors(1000s)
\item Symmetrical: Each processor has a full copy of the operating system
\item Asymmetrical: Only one processor carries the full operating system, Other processors are operated by a small operating system stub to transfer code and tasks
\end{itemize}
\item Massively parallel systems are a niche market and hence parallel operating systems are usually very specific to the hardware design and application area.
\end{itemize}
\subsubsection{Distributed Operating Systems}
\label{sec:org4ce5a5a}
\begin{itemize}
\item Characteristics:
\begin{itemize}
\item Support for a medium number of loosely coupled processors
\item Processors execute a small operating system kernel providing essential communication services
\item Other operating system services are distributed over available processors
\item Services can be replicated in order to improve scalability and availability
\item Distribution of tasks and data transparent to users (single system image)
\end{itemize}
\item Examples:
\begin{itemize}
\item Amoeba (Vrije Universiteit Amsterdam)
\item Plan 9 (Bell Labs, AT\&T)
\end{itemize}
\end{itemize}
\subsubsection{Real-time Operating Systems}
\label{sec:org0e4d392}
\begin{itemize}
\item Characteristics:
\begin{itemize}
\item Predictability
\item Logical correctness of the offered services
\item Timeliness of the offered services
\item Services are to be delivered not too early, not too late
\item Operating system executes processes to meet time constraints
\end{itemize}
\item Examples:
\begin{itemize}
\item QNX
\item VxWorks
\item RTLinux, RTAI, Xenomai
\item Windows CE
\end{itemize}
\end{itemize}
\subsubsection{Embedded Operating Systems}
\label{sec:org19a0ac2}
\begin{itemize}
\item Characteristics:
\begin{itemize}
\item Usually real-time systems, sometimes hard real-time systems
\item Very small memory footprint (even today!)
\item No or limited user interaction
\item 90-95 \% of all processors are running embedded operating systems
\end{itemize}
\item Examples:
\begin{itemize}
\item Embedded Linux, Embedded BSD
\item Symbian OS, Windows Mobile, iPhone OS, BlackBerry OS, Palm OS
\item Cisco IOS, JunOS, IronWare, Inferno
\item Contiki, TinyOS, RIOT
\end{itemize}
\end{itemize}
\subsection{evolution of os}
\label{sec:orgb38d012}
\begin{itemize}
\item Vacuum Tubes: no operating system
\item Transistors: batch systems automatically process job queues
\item Integrated Circuit: Spooling (Simultaneous Peripheral Operation On Line), Multiprogramming and Time-sharing
\item VLSI: Personal computer (CP/M, MS-DOS, Windows, Mac OS, Unix), Network operating systems (Unix), Distributed operating systems (Amoeba, Mach, V)
\end{itemize}
\subsection{Monolithic and Modular Operating Systems}
\label{sec:orgede1d68}
\begin{itemize}
\item Example: Linux
\item Modules can be platform independent
\item Easier to maintain and to develop
\item Increased reliability / robustness
\item All services are in the kernel with the same privilege level
\item May reach high efficiency Example: Linux
\end{itemize}
\subsection{Monolithic and Layered Operating Systems}
\label{sec:org87d9aec}
\begin{itemize}
\item Easily portable, significantly easier to maintain
\item Often reduced efficiency because of the need to go through many layered interfaces
\item Rigorous implementation of the stacked virtual machine perspective
\item Services offered by the various layers are important for the overall performance
\item Example: THE (Dijkstra, 1968)
\end{itemize}
\subsection{Virtual Machines}
\label{sec:orgb37e637}
\begin{itemize}
\item Virtualization of the hardware
\item Multiple operating systems can execute concurrently
\item Separation of multi-programming from other operating system services
\item Examples: IBM VM/370 (’79), VMware (1990s), XEN (2003)
\end{itemize}
\subsection{micro kernel}
\label{sec:orgd529a03}
\begin{itemize}
\item bring up the driver, just implement the minimum of a kernel
\end{itemize}
\section{Hardware}
\label{sec:org5fb92e9}
\subsection{Common Computer Architecture}
\label{sec:org2ceae66}
\begin{itemize}
\item sequencer
\item ALU
\item Registers
\item buses:
\begin{itemize}
\item control bus
\item address bus
\item data bus
\end{itemize}
\end{itemize}
\subsubsection{CPU registers}
\label{sec:orgfc0990e}
\begin{itemize}
\item Typical set of registers:
\begin{itemize}
\item Processor status register
\item Instruction register (current instruction)
\item Program counter (current or next instruction) - Stack pointer (top of stack)
\item Special privileged registers
\item Dedicated registers
\item Universal registers
\end{itemize}
\item Privileged registers are only accessible when the processor is in privileged mode
\item Switch from non-privileged to privileged mode via traps or interrupts
\end{itemize}
\subsubsection{}
\label{sec:org77de38c}
\subsection{I/O Systems and Interrupts}
\label{sec:org624e4c3}
\subsubsection{I/O Devices}
\label{sec:org3dab95f}
\begin{itemize}
\item I/O devices are essential for every computer
\item Typical classes of I/O devices:
\begin{itemize}
\item clocks, timers
\item user-interface devices
\item document I/O devices (scanner, printer, \ldots{})   - audio and video equipment
\item network interfaces
\item mass storage devices
\item sensors and actuators in control applications
\end{itemize}
\item Device drivers are often the biggest component of general purpose operating system kernels
\end{itemize}
\subsubsection{Basic I/O Programming}
\label{sec:orgabd7e4c}
\begin{itemize}
\item \textbf{Status driven}: the processor polls an I/O device for information
\begin{itemize}
\item Simple but inefficient use of processor cycles
\end{itemize}
\item \textbf{Interrupt driven}: the I/O device issues an interrupt when data is available or an I/O operation has been completed
\begin{itemize}
\item Program controlled: Interrupts are handled by the processor directly
\item Program initiated: Interrupts are handled by a DMA-controller and no processing is performed by the processor (but the DMA transfer might steal some memory access
\end{itemize}
cycles, potentially slowing down the processor)
\begin{itemize}
\item Channel program controlled: Interrupts are handled by a dedicated channel device, which is usually itself a micro-processor
\end{itemize}
\end{itemize}
\subsubsection{Interrupts}
\label{sec:org014d943}
\begin{itemize}
\item Interrupts can be triggered by hardware and by software
\begin{itemize}
\item clock interrupts
\item keyboard
\item timer
\end{itemize}
\item Interrupt control:
\begin{itemize}
\item grouping of interrupts
\item encoding of interrupts
\item prioritizing interrupts
\item enabling / disabling of interrupt sources
\end{itemize}
\item Interrupt identification:
\begin{itemize}
\item interrupt vectors, interrupt states
\end{itemize}
\item Context switching:
\begin{itemize}
\item mechanisms for CPU state saving and restoring
\end{itemize}
\end{itemize}
\subsubsection{Interrup Servie Routines}
\label{sec:orgd703584}
\begin{itemize}
\item minimal hardware support (supplied by the CPU)
\begin{itemize}
\item save essential CPU registers
\item jump to the vectorized interrupt servie routine
\item restore essential CPU register to return
\end{itemize}
\item minimal wrapper (supplied by the OS)
\begin{itemize}
\item Save remaining CPU registers
\item Save stack-frame
\item Execute interrupt service code
\item Restore stack-frame
\item Restore CPU registers
\end{itemize}
\end{itemize}
\subsection{Memory}
\label{sec:orgb22ed16}
\subsubsection{memory size and access time}
\label{sec:org015a105}
\begin{itemize}
\item trade-off between memory speed and memory size
\begin{itemize}
\item register is fast, but super small
\item L1, L2, L3 cache increases in size, decrease in speed
\item main memory, big
\item disk
\end{itemize}
\end{itemize}
\subsubsection{memory segments}
\label{sec:org7e61dcb}
\begin{table}[htbp]
\caption{Different Segments in Memory}
\centering
\begin{tabular}{ll}
Segments & Description\\
\hline
text & machine instructions\\
data & static variable and constants, may be further devided into initialized and uninitialized data\\
heap & dynamically allocated data structures\\
stack & automatically allocated local variables, management of function calls\\
\end{tabular}
\end{table}
\begin{itemize}
\item Memory used by a program is usually partitioned into different segments that serve different purposes and may have different access rights
\end{itemize}
\subsubsection{stack frame}
\label{sec:orgf91510d}
\begin{itemize}
\item arguments pushed onto the stack in a reverse order
\item buttom is arguments for the function calls
\item then return address(instruction pointer)
\item frame pointer
\item stack locals
\end{itemize}
\subsubsection{stack smashing attack}
\label{sec:org12ea5c6}
\begin{itemize}
\item unchecked boundry will lead to stack overflow
\end{itemize}
\subsubsection{caching}
\label{sec:orgffb58ca}
\begin{itemize}
\item caching is A general technique to speed up memory access by introducing smaller and faster memories which keep a copy of frequently / soon needed data
\item cache hit: A memory access which can be served from the cache memory
\item cache miss: A memory access which cannot be served from the cache and requires access to slower memory
\item cache write through: A memory update which updates the cache entry as well as the slower memory cell
\item Delayed write: A memory update which updates the cache entry while the slower memory cell is updated at a later point in time
\end{itemize}
\subsubsection{Locality}
\label{sec:org5e0a0f7}
\begin{itemize}
\item Cache performance is relying on:
\begin{itemize}
\item Spatial locality: Nearby memory cells are likely to be accessed soon
\item Temporal locality: Recently addressed memory cells are likely to be accessed again soon
\end{itemize}
\item Iterative languages generate linear sequences of instructions (spatial locality)
\item Functional / declarative languages extensively use recursion (temporal locality)
\item CPU time is in general often spend in small loops/iterations (spatial and temporal locality)
\item Data structures are organized in compact formats (spatial locality)
\end{itemize}
\section{Processes and Thread}
\label{sec:org1b06491}
\subsection{Fundamental Concepts}
\label{sec:orgf7ea0a7}
\subsubsection{Separation of Mechanisms and Policies}
\label{sec:org4c30b8a}
\begin{itemize}
\item An important design principle is the separation of policy from mechanism.
\item Mechanisms determine how to do something.
\item Policies decide what will be done.
\item The separation of policy and mechanism is important for flexibility, especially since policies are likely to change.
\end{itemize}
\subsubsection{User Mode}
\label{sec:org6996b62}
\begin{itemize}
\item the processor executes machine instructions of (user space) processes;
\item the instruction set of the processor is restricted to the so called unprivileged instruction set;
\item the set of accessible registers is restricted to the so called unprivileged register set;
\item the memory addresses used by a process are typically mapped to physical memory addresses by a memory management unit;
\item direct access to hardware components is protected by using hardware protection where possible;
\item direct access to the state of other concurrently running processes is restricted.
\end{itemize}
\subsubsection{System Mode}
\label{sec:org4836360}
\begin{itemize}
\item the processor executes machine instructions of the operating system kernel;
\item all instructions of the processor can be used, the so called privileged instruction set;
\item all registers are accessible, the so called privileged register set;
\item direct access to physical memory addresses and the memory address mapping tables is enabled;
\item direct access to the hardware components of the system is enableds;
\item the direct manipulation of the state of processes is possible.
\end{itemize}
\subsubsection{Entering the Operating System Kernel}
\label{sec:orga49df96}
\begin{itemize}
\item System calls
\begin{itemize}
\item Synchronous to the running process
\item Parameter transfer via registers, the call stack or a parameter block
\end{itemize}
\item Hardware traps
\begin{itemize}
\item Synchronous to a running process (devision by zero)
\item Forwarded to a process by the operating system
\end{itemize}
\item Hardware interrupts
\begin{itemize}
\item Asynchronous to the running processes
\item Call of an interrupt handler via an interrupt vector
\end{itemize}
\item Software interrupts
\begin{itemize}
\item Asynchronous to the running processes
\end{itemize}
\end{itemize}
\subsection{Processes}
\label{sec:orge3d4cd7}
\subsubsection{Characteristics}
\label{sec:org4d62b43}
\begin{itemize}
\item an instance of a program under execution
\item A process uses/owns resources (CPU, memory, file) and is charactererized by the following:
\begin{enumerate}
\item A sequence of machine instructions which determines the behavior of the running program (control flow)
\item The current state of the process given by the content of the processor’s registers, the contents of the stack, and the contents of the heap (internal state)
\item The state of other resources (e.g., open files or network connections, timer, devices) used by the running program (external state)
\end{enumerate}
\item Processes are sometimes also called tasks.
\end{itemize}
\subsubsection{State Machine View of Processes}
\label{sec:orga74aea9}
\begin{itemize}
\item new: just created
\item ready: ready to running
\item running: executing
\item blocked: not ready to run, wait for resources
\item terminated: just finished, not yet removed
\end{itemize}
\subsubsection{Queueing Model View of Processes}
\label{sec:org9fc01eb}
\begin{itemize}
\item Processes are enqueued if resources are not readily available or if processes wait for events
\item Dequeuing strategies have strong performance impact
\item Queueing models can be used for performance analysis
\end{itemize}
\subsubsection{Process Control Block}
\label{sec:orgec41595}
\begin{itemize}
\item process are internally represented by a process control block (PCB)
\begin{itemize}
\item process identification
\item process state
\item saved registers during context switches
\item scheduling information
\item assigned memory regions
\item open files or network connections
\item accounting info (\# of CPU cycles, IO operations)
\item pointers to other PCBs
\end{itemize}
\end{itemize}
\subsubsection{Process Lists}
\label{sec:org71388a5}
\begin{itemize}
\item PCBs are often organized in doubly-linked lists or tables
\item can be queued by pointer operations
\item run queue length of the CUP is a good load indicator
\item The system load often defined as the exponentially smoothed average of the run queue length over 1, 5 and 15 minutes
\begin{itemize}
\item usually count when there is a interrupt
\end{itemize}
\end{itemize}
\subsubsection{Process Creation}
\label{sec:org8ee2216}
\begin{itemize}
\item The fork() system call creates a new child process
\begin{itemize}
\item which is an exact copy of the parent process
\item except that the result of the system call differs
\end{itemize}
\item exec() system call replaces the current process image with a new process image
\end{itemize}
\subsubsection{Process Tree}
\label{sec:org8d9c0cc}
\begin{itemize}
\item the first process is created when the system is initialized (init / systemd)
\item All other processes are created using fork(), which leads to a process tree
\item PCBs often contain pointers to parent PCBs
\end{itemize}
\subsubsection{Process Termination}
\label{sec:org621a66a}
\begin{itemize}
\item Processes can terminate themself by calling exit()
\item The wait() system call allows processes to wait for the termination of a child process
\item Terminating processes return a numeric result code
\item shell application:
\begin{itemize}
\item first fork
\item in the child we execute the operation
\item and the parent wait() for the child
\end{itemize}
\end{itemize}
\subsubsection{POSIX API (fork, exec)}
\label{sec:orga576998}
\begin{verbatim}
#include <unistd.h>
pid_t getpid(void);
pid_t getppid(void);
pid_t fork(void);
int execve(const char  *filename, char *const argv [],
           char *const envp[]);
extern char **environ;
int execl(const char *path, const char *arg, ...);
int execlp(const char *file, const char *arg, ...);
int execle(const  char  *path,  const  char  *arg, ...,
           char * const envp[]);
int execv(const char *path, char *const argv[]);
int execvp(const char *file, char *const argv[]);
\end{verbatim}
\subsubsection{POSIX API (wait, exit)}
\label{sec:org1903395}
\begin{verbatim}
#include <stdlib.h>
void exit(int status);
int atexit(void (*function)(void));
#include <unistd.h>
void _exit(int status);
pid_t wait(int *status);
pid_t waitpid(pid_t pid, int *status, int options); // wait for a process
#include <sys/time.h>
#include <sys/resource.h>
#include <sys/wait.h>
pid_t wait3(int *status, int options, struct rusage *rusage);
pid_t wait4(pid_t pid, int *status, int options, struct rusage *rusage);
\end{verbatim}
\end{document}